\chapter{Realizacja projektu}


\section{Wykorzystane technologie}

\subsection{Język programowania Python}

Kod projektu został napisany w języku Python. Wybór tego języka był spowodowany szeroką dostępnością bibliotek i
narzędzi wspomagających uczenie maszynowe. W szczególności należy zwrócić uwagę na biblioteki \textit{Tensorflow} oraz \textit{PyTorch}.
Są to dwie najpopularniejsze biblioteki do operowania na tensorach (tensor to uogólnienie macierzy na wiele wymiarów).
Obie pozwalają na przyśpieszanie obliczeń z użyciem zewnętrznych akceleratorów takich jak np. karty graficzne.

Podczas prac badawczych, wykorzystana wersja Pythona to \textit{3.12.0}.

\subsection{Biblioteka PyTorch}

Biblioteka PyTorch jest obecnie najpopularniejszą biblioteką uczenia maszynowego.
Pozwala ona projektować obliczenia w formie modułów i posiada silnik automatycznego różniczkowania grafu obliczeniowego (z ang. \textit{autograd}).
Owy silnik jest kluczowy z perspektywy trenowania sieci neuronowych, ponieważ jest w stanie optymalizować parametry modelu.
Warto zwrócić uwagę na to, że PyTorch stawia nacisk na przejrzystość obliczeń - dla porównania operacje w Tensorflow są nierzadko
ukryte pod gotowymi funkcjami (zob. listing \ref{lst:tensorflow_sample} i \ref{lst:pytorch_sample})

\begin{lstlisting}[language=ipython,caption={Przykładowa sieć neuronowa w Tensorflow},label={lst:tensorflow_sample}]
tensorflow_model = tf.keras.Sequential([
    tf.keras.layers.Flatten(input_shape=(28, 28)),
    tf.keras.layers.Dense(128, activation='relu'),
    tf.keras.layers.Dense(10, activation='softmax')
])

tensorflow_model.compile() # Brak doglebnej kontroli nad przeplywem obliczen
\end{lstlisting}


\begin{lstlisting}[language=ipython,caption={Przykładowa sieć neuronowa w PyTorch},label={lst:pytorch_sample}]
class PyTorchNetwork(nn.Module):
    def __init__(self):
        super(Net, self).__init__()
        self.flatten = nn.Flatten()
        self.fc1 = nn.Linear(28 * 28, 128)
        self.relu = nn.ReLU()
        self.fc2 = nn.Linear(128, 10)

    def forward(self, x): # Dokladna kontrola nad przeplywem obliczen
        x = self.flatten(x)
        x = self.fc1(x)
        x = self.relu(x)
        x = self.fc2(x)
        return x
\end{lstlisting}

PyTorch oferuje również zestaw narzędzi do przetwarzania danych, dostarcza również gotowe, pretrenowane modele dla najpopularniejszych architektur.
Przykładami są pakiety \textit{torchvision} lub \textit{torch.utils}. Cała biblioteka jest w stanie wykorzystywać karty graficzne (z ang. \textit{graphical processing unit, GPU}) do przyspieszania obliczeń.

\subsection{Biblioteka OpenCV}

Biblioteka OpenCV dostarcza implementacje wielu konwencjonalnych algorytmów wizji komputerowej.
OpenCV eksponuje interfejs programistyczny, dzięki któremu programista Pythona może komunikować się z implementacją algorytmów w C++.
To zapewnia szybkość wykonywania obliczeń, jednocześnie nie wymuszając na programiście pisania kodu w C++. Przykład wykorzystania biblioteki OpenCV widoczny jest na listingu \ref{lst:opencv_sample}
 - kod wylicza momenty Hu.

OpenCV w niniejszej pracy jest wykorzystywane do automatycznej ekstrakcji kwadratów z komórkami na podstawie dużego zdjęcia spod mikroskopu.
Więcej informacji na ten temat w rozdziale \ref{sec:automatyczny-licznik-limfoblastow}.

\begin{lstlisting}[language=ipython,caption={Obliczenie momentów Hu z użyciem OpenCV}, label={lst:opencv_sample}]
import cv2

image = cv2.imread('obraz.jpg', 0)  # 0 oznacza wczytanie obrazu w odcieniach szarosci

hu_moments = cv2.HuMoments(cv2.moments(image)).flatten()

print("Momenty Hu:")
for i in range(0, 7):
    print(f"Moment {i+1}: {hu_moments[i]}")
\end{lstlisting}

\subsection{Sprzęt}

Do prac badawczych został użyty Apple Macbook Pro 2020 (M1/16GB pamięci). Trenowanie sieci odbywało się na platformie \textit{kaggle.com}, która oferuje bezpłatne 30 godzin obliczeń co miesiąc z kartą graficzną \textit{NVIDIA P100}.

\section{Źródło danych}

\section{Struktura projektu}

\subsection{Przygotowanie danych}

\subsection{Trening sieci neuronowej}

\subsection{Ocena jakości modelu}


\section{Interfejs użytkownika}


\section{Automatyczny licznik limfoblastów}\label{sec:automatyczny-licznik-limfoblastow}

