\chapter{Wstęp}

\section{Wprowadzenie}

%TODO Wprowadzenie

\section{Motywacja}

W ostatnich latach obserwuje się coraz większe zastosowanie informatyki, a w szczególności algorytmów uczenia maszynowego,
w medycynie i diagnostyce.
Jednym z ważniejszych zastosowań jest wykorzystanie widzenia komputerowego, wspieranego sieciami neuronowymi, do przetwarzania i analizy obrazów medycznych.
Przykładowo odpowiedni system może analizować wyniki tomografii komputerowej bądź rezonansu magnetycznego, by stwierdzać niepokojące zmiany w ciele człowieka.
Odpowiednia sieć neuronowa widzenia komputerowego może na przykład badać skany ludzkiego mózgu i ustalać, czy na zdjęciu jest widoczny guz \cite{brain_tumor}.
Innym przykładem jest traktowanie zapisu sygnału elektrokardiogramu jako szerokiego
obrazu \cite{ecg_cnn}.
Wtedy program komputerowy może oznaczać fragmenty, na które lekarz powinien zwrócić szczególną uwagę.

W niniejszej pracy zaproponowany algorytm uczenia maszynowego jest w stanie rozpoznawać komórki w szpiku kostnym.
Klasyfikowanie typów komórek jest kluczowe z perspektywy diagnostyki wielu chorób.
Przykładowo stwierdzenie ostrej białaczki limfoblastycznej opiera się na m.in.
zliczeniu 1000 komórek i policzeniu, ile z nich jest limfoblastami \cite{blast_counting_diagnosis}.
Obecnie nie istnieje w pełni automatyczny system ich zliczania, więc w procesie diagnostyki musi być zaangażowany człowiek.
Postęp w zakresie

\section{Cele i zakres pracy}

Cele i zakres pracy
