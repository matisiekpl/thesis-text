\chapter{Wstęp}


\section{Wprowadzenie}

Rozwój technologii informatycznych, w szczególności uczenia maszynowego, otwiera nowe możliwości w wielu dziedzinach nauki i przemysłu.
Jednym z obszarów życia, w którym te technologie mogą odnieść duży sukces, jest medycyna.
Zastosowanie komputerów do analizy danych medycznych \textcolor{red}{może wpłynąć pozytywnie na proces} leczenia wielu chorób.
Sztuczna inteligencja daje możliwość zautomatyzowania czasochłonnych zadań w diagnostyce i zaoszczędzenia wielu godzin pracy lekarza diagnosty.

Celem niniejszej pracy jest zastosowanie splotowych sieci neuronowych do klasyfikacji komórek szpiku kostnego na podstawie zdjęć rozmazów.
Wykorzystanie tej technologii może znacznie przyspieszyć i ułatwić proces diagnozy, co jest kluczowe dla skutecznego leczenia wielu chorób takich \textcolor{red}{jak na przykład} nowotwory krwi.

Praca ma charakter badawczy i jest próbą zastosowania nowoczesnych technologii informatycznych w praktycznym problemie medycznym.


\section{Motywacja}

W ostatnich latach obserwuje się coraz \textcolor{red}{częstsze} zastosowanie informatyki, a w szczególności algorytmów uczenia maszynowego,
w medycynie i diagnostyce.
Jednym z ważniejszych zastosowań \textcolor{red}{tych metod} jest wykorzystanie widzenia komputerowego, wspieranego sieciami neuronowymi, do przetwarzania i analizy obrazów medycznych.
Przykładowo, odpowiedni system może analizować wyniki tomografii komputerowej bądź rezonansu magnetycznego, by stwierdzić niepokojące zmiany w ciele człowieka.
Odpowiednia sieć neuronowa widzenia komputerowego może badać skany ludzkiego mózgu i ustalać, czy na zdjęciu jest widoczny guz \cite{brain_tumor}.
Innym przykładem jest traktowanie zapisu sygnału elektrokardiogramu jako szerokiego
obrazu \cite{ecg_cnn}.
Wtedy program komputerowy może oznaczać fragmenty, na które lekarz powinien zwrócić szczególną uwagę.

W niniejszym projekcie został \textcolor{red}{zaproponowany algorytm uczenia maszynowego, którego zadaniem jest rozpoznawać} rodzaje komórek w szpiku kostnym.
Klasyfikowanie typów komórek jest kluczowe z perspektywy diagnostyki wielu chorób.
Przykładowo, stwierdzenie ostrej białaczki limfoblastycznej opiera się na m.in.
\textcolor{red}{przeglądnięciu ok. 1000 komórek pochodzących z krwi i stwierdzeniu}, ile z nich jest limfoblastami~\cite{blast_counting_diagnosis}.
Obecnie nie istnieje w pełni automatyczny system ich zliczania, więc w procesie diagnostyki musi być zaangażowany człowiek.
Postęp w tym zakresie oznaczałby znaczne przyspieszenie diagnozy pacjenta, a co za tym idzie polepszenie rokowania wyleczenia.


\section{Cele i zakres pracy}

Celem pracy jest porównanie kilku algorytmów uczenia maszynowego z wykorzystaniem splotowych sieci neuronowych do rozpoznawania typów komórek na podstawie \textcolor{red}{zdjęć przedstawiających obraz mikroskopowy}.
Zdjęcia przedstawiają obraz mikroskopowy rozmazu szpiku kostnego. Wykorzystany zbiór danych pochodzi z
Monachijskiego Labolatorium Białaczek (\textit{MLL Münchner Leukämielabor} \cite{mll}) i jest dostępny na platformie \textit{kaggle.com} \cite{dataset}.
Praca ma za zadanie porównać różne architektury splotowych sieci neuronowych i stwierdzić, która z nich daje najlepsze \textcolor{red}{rezultaty pod względem precyzji i liczby pomyłek w klasyfikacji}.

Praca zawiera także omówienie algorytmu wstępnego przetwarzania danych (z ang. \textit{preprocessing}) oraz algorytmu ekstrakcji obrazów komórek z dużego skanu rozmazu.
