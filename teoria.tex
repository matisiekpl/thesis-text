\chapter{Zagadnienia teoretyczne}

\section{Sieci neuronowe}

Sieci neuronowe stanowią jedną z najpopularniejszych technik uczenia maszynowego. Są one inspirowane działaniem ludzkiego mózgu. Sieci składają się z neuronowów, pogrupowanych w połączone ze sobą warstwy.

Ogólna architektura sieci neuronowej:
\begin{itemize}
    \item Warstwa wejściowa - ta warstwa przyjmuje dane początkowe w formie tensora liczb rzeczywistych. Danymi wejściowymi mogą być poziomy jasności obrazu, dane o cenach lub sygnale (w przypadku prognozowania serii czasowych) lub osadzenia (z ang \textit{embeddings}) dla modeli językowych.
    \item Warstwy ukryte - te warstwy przetwarzają dane wejściowe przez sekwencję transformacji matematyczncyh. Neurony są połączone z poprzednią warstwą, a siła ich połączenia jest wyrażana przez tak zwane wagi. Wartość, którą przyjmie dany neuron jest liczona na podstawie ważonej sumy wartości neuronów z poprzedniej warstw. Po obliczeniu sumy, stosowana jest funkcja aktywacji. Ma ona za zadanie wprowadzić nieliniowość obliczeń, która jest wymagana do rozpoznawania zaawansowanych wzorców.
    \item Warstwa wyjściowa - warstwa, która generuje końcowy tensor. Tensor może zawierać informacje takie jak rozkład prawdopodobieństwa przyporządkowania do poszczególnych klas. Może również zawierać prognozowaną liczbę, w przypadku modeli regresyjnych.
\end{itemize}

Proces trenowania sieci neuronowej polega na modyfikowaniu wag pomiędzy połączeniami poszczególnych neuronów tak, by minimalizować różnicę między tensorem wyjściowym, a tensorem oczekiwanym.
Najczęściej stosuje się do tego zadania algorytmy optymalizacyjne oparte na pochodnych i gradientach. Przykładem może być \textit{algorytm spadku wzdłuż gradientu}.
W rzeczywistych zadaniach stosuje się jednak bardziej wyszukane algorytmy takie jak \textit{algorytm stochastycznego spadku wzdłuż gradientu} albo \textit{algorytm Adam}.

Sieci neuronowe okazały się dużym sukcesem w wielu trudnych zadaniach uczenia maszynowego.
Są z powodzeniem stosowane w takich branżach jak medycyna, rozrywka, cyberbezpieczeństwo czy motoryzacja. Na przykład zespół Tesla Vision trenuje duże sieci neuronowe by stworzyć pojazd autonomiczny, zdolny poruszać się bezpiecznie bez kierowcy.
Sieci neuronowe coraz częściej będą wykonywać określone zadania szybciej i skuteczniej niż człowiek.
Czas reakcji systemu autopilot samochodu autonomicznego jest znacznie krótszy niż czas reakcji kierowcy, co w długiej perspektywie może przyczynić się do zwiększenia bezpieczeństwa na drogach i szlakach komunikacyjnych.
Innym bardzo ważnym zastosowaniem sieci neuronowych jest medycyna. Naukowcy z Google DeepMind stworzyli model \textit{AlphaFold}, który jest w stanie przewidywać molekularną strukturę białek.
Postęp informatyki w naukach biologicznych i chemicznych otwiera perspektywę na leczenie chorób, które dotychczas były nieuleczalne i umożliwia lepsze zrozumienie organizmu człowieka.

\section{Splotowe sieci neuronowe}

\section{Architektury sieci neuronowych}

\section{Problem klasyfikacji komórek krwi w rozmazach szpiku kostnego}

\section{Konwencjonalne algorytmy widzenia komputerowego}

\section{GradCAM - wytłumaczalne uczenie maszynowe}