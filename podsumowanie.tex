\chapter{Podsumowanie}

W niniejszym projekcie inżynierskim omówiono realizację systemu klasyfikatora rodzajów komórek szpiku kostnego na podstawie uczenia maszynowego.
Porównane zostały różne architektury splotowych sieci neuronowych.
Architekturą, która daje najlepsze rezultaty okazała się \textbf{EfficientNet B4}.
Przedstawione rozwiązanie oferuje podobną jakość klasyfikacji (uwzględniając ważoną miarę F1) jak opublikowane wcześniej prace naukowe.
Warto zauważyć jednak, że sieć EfficientNet B4 dostarcza podobnej jakości predykcji jak omawiana w innych artykułach architektura \textit{ResNeXt}, pomimo znacznie mniejszej ilości parametrów do optymalizacji (19 milionów parametrów w przypadku EfficientNet B4 versus 25 milionów w przypadku ResNeXt \cite{resnext}).

Opisano również algorytm ekstrakcji obrazów komórek z dużego skanu rozmazu, dzięki któremu można automatycznie wyznaczyć dane wejściowe do sieci neuronowej (sieć neuronowa wymaga obrazu pojedynczej komórki).
Do jego opracowania wykorzystano takie metody widzenia komputerowego jak progowanie, znajdowanie konturów czy momenty Hu.

W ramach projektu stworzono również przejrzysty interfejs użytkownika, który umożliwia użytkownikowi łatwą interakcję z modelem sztucznej inteligencji.
Po wysłaniu obrazu komórki na serwer pojawia się tabela rozkładu prawdopodobieństwa wraz z wizualizacją mapy cieplnej \textit{GradCam}.
Mapa cieplna pozwala użytkownikowi lepiej zrozumieć proces predykcji klas.

Podsumowując, zaprezentowany system stanowi przykład praktycznego zastosowania uczenia maszynowego i splotowych sieci neuronowych w medycynie.
W świetle obecnego tempa rozwoju technik sztucznej inteligencji, można się spodziewać, że w przyszłości rozwiązania oparte o SI będą rozwiązywały coraz bardziej złożone problemy w wielu branżach i dziedzinach życia.
Szeroko pojęta automatyzacja, napędzana przez sztuczną inteligencję, będzie prowadziła do przyspieszenia różnych procesów oraz oszczędności wielu godzin pracy specjalistów.