\chapter{Podsumowanie}

\section{Potencjalne możliwości rozwoju projektu}

Niniejszy projekt posiada potencjał na dalszy rozwój. Kluczowe ścieżki dalszych badań nad zagadnieniem dotyczyłyby:
\begin{itemize}
\item dokonania dalszej optymalizacji sieci neuronowych, które okazały się skuteczne dla rozwiązania postawionego w pracy problemu. Ze względu na ograniczone zasoby obliczeniowe porównanie różnej ilości warstw splotowych było utrudnione - trening wymagałby kilkakrotnie więcej godzin pracy procesora graficznego. Hiperparametrami do sprawdzenia byłyby wtedy zmienne takie jak: typ warstw aktywacji, ilość i rozmiar filtrów, głębokość całej sieci.
\item szerszego sprawdzenia efektów normalizacji obrazów wejściowych. Obecnie sposób normalizacji jasności i kontrastu obrazu jest stały. Dynamiczne modyfikowanie metody procesowania danych wejściowych mogłoby spowodować poprawę jakości predykcji tej samej sieci neuronowej.
\item opracowania pętli informacji zwrotnej pochodzącej od człowieka. Doświadczony diagnosta stwierdzając, że model zwrócił nieprawidłową nazwę klasy, mógłby oznaczać region na którym sieć neuronowa powinna w szczególności się skupić.
\end{itemize}

\section{Zakończenie}

W niniejszym projekcie inżynierskim omówiono realizację systemu klasyfikującego rodzaje komórek szpiku kostnego za pomocą uczenia maszynowego.
Porównane zostały rezultaty trenowania różnych architektur splotowych sieci neuronowych.
Architekturą, która daje najlepsze rezultaty okazała się \textbf{EfficientNet B4}.
Przedstawione rozwiązanie oferuje podobną jakość klasyfikacji (na podstawie ważonej miary F1) jak opublikowana wcześniej praca naukowa autorstwa dostawcy danych.
Warto zauważyć jednak, że sieć EfficientNet B4 dostarcza podobnej jakości predykcji jak omawiana w innych artykułach architektura \textit{ResNeXt}, pomimo znacznie mniejszej ilości parametrów do optymalizacji (19 milionów parametrów w przypadku EfficientNet B4 w stosunku do 25 milionów w przypadku ResNeXt \cite{resnext}).

Opisano również algorytm ekstrakcji obrazów komórek z dużego skanu rozmazu, dzięki któremu można automatycznie wyznaczyć dane wejściowe do sieci neuronowej (sieć neuronowa wymaga obrazu pojedynczej komórki).
Do jego opracowania wykorzystano takie metody przetwarzania obrazów jak progowanie, znajdowanie konturów czy momenty Hu.

W ramach projektu stworzono również przejrzysty interfejs użytkownika, który umożliwia użytkownikowi łatwą interakcję z modelem sztucznej inteligencji.
Po wysłaniu obrazu komórki na serwer pojawia się tabela rozkładu prawdopodobieństwa rozpoznania poszczególnych klas wraz z wizualizacją mapy cieplnej \textit{GradCam}.
Mapa cieplna pozwala użytkownikowi lepiej zrozumieć proces predykcji klas.

Podsumowując, zaprezentowany system stanowi przykład praktycznego zastosowania uczenia maszynowego i splotowych sieci neuronowych w medycynie.
W świetle obecnego tempa rozwoju technik sztucznej inteligencji, można się spodziewać, że w przyszłości rozwiązania oparte o SI będą rozwiązywały coraz bardziej złożone problemy w wielu branżach i dziedzinach życia.
Szeroko pojęta automatyzacja, napędzana przez sztuczną inteligencję, będzie prowadziła do przyspieszenia różnych procesów oraz oszczędności wielu godzin pracy specjalistów.