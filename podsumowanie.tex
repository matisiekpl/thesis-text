\chapter{Podsumowanie}

W niniejszym projekcie inżynierskim omówiono realizację systemu \textcolor{red}{klasyfikującego} rodzaje komórek szpiku kostnego \textcolor{red}{za pomocą} uczenia maszynowego.
Porównane zostały \textcolor{red}{rezultaty trenowania różnych architektur} splotowych sieci neuronowych.
Architekturą, która daje najlepsze rezultaty okazała się \textbf{EfficientNet B4}.
Przedstawione rozwiązanie oferuje podobną jakość klasyfikacji (\textcolor{red}{na podstawie} ważonej miary F1) jak \textcolor{red}{opublikowana wcześniej praca naukowa autorstwa dostawcy danych}.
Warto zauważyć jednak, że sieć EfficientNet B4 dostarcza podobnej jakości predykcji jak omawiana w innych artykułach architektura \textit{ResNeXt}, pomimo znacznie mniejszej ilości parametrów do optymalizacji (19 milionów parametrów w przypadku EfficientNet B4 w \textcolor{red}{stosunku do} 25 milionów w przypadku ResNeXt \cite{resnext}).

Opisano również algorytm ekstrakcji obrazów komórek z dużego skanu rozmazu, dzięki któremu można automatycznie wyznaczyć dane wejściowe do sieci neuronowej (sieć neuronowa wymaga obrazu pojedynczej komórki).
Do jego opracowania wykorzystano takie metody \textcolor{red}{przetwarzania obrazów} jak progowanie, znajdowanie konturów czy momenty Hu.

W ramach projektu stworzono również przejrzysty interfejs użytkownika, który umożliwia użytkownikowi łatwą interakcję z modelem sztucznej inteligencji.
Po wysłaniu obrazu komórki na serwer pojawia się tabela rozkładu prawdopodobieństwa \textcolor{red}{ rozpoznania poszczególnych klas} wraz z wizualizacją mapy cieplnej \textit{GradCam}.
Mapa cieplna pozwala użytkownikowi lepiej zrozumieć proces predykcji klas.

Podsumowując, zaprezentowany system stanowi przykład praktycznego zastosowania uczenia maszynowego i splotowych sieci neuronowych w medycynie.
W świetle obecnego tempa rozwoju technik sztucznej inteligencji, można się spodziewać, że w przyszłości rozwiązania oparte o SI będą rozwiązywały coraz bardziej złożone problemy w wielu branżach i dziedzinach życia.
Szeroko pojęta automatyzacja, napędzana przez sztuczną inteligencję, będzie prowadziła do przyspieszenia różnych procesów oraz oszczędności wielu godzin pracy specjalistów.